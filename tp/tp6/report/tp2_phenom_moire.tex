\documentclass[a4paper,10pt]{article}
\input{/Users/WannaGetHigh/workspace/latex/macros.tex}

\title{TP 2: Att�nuation du ph�nom�ne de moir� version 2012}
\author{Fran�ois \bsc{Lepan}}

\begin{document}
\maketitle

\section{Manipulations}

Tout d'abord on charge l'image "256a.jpg" (\emph{cf.}~Fig.~\ref{fig:256a})

\begin{figure}[ht]
\begin{center}
	\includegraphics[width=10cm]{../256_a.jpg}
\end{center}
	\caption{Image 256\_a.jpg}
	\label{fig:256a}
\end{figure}

\subsection{Fr�quence spatiale $\omega$(cycle/pixel) du motif cyclique de la Fig.~\ref{fig:256a} }

Gr�ce a l'outils \emph{scolling tool} on peut r�cup�rer la p�riode spatiale T qui est de 5 pixels.\\
Or la fr�quence spatiale F est l'inverse de la p�riode spatiale. \\
Donc F = $\frac{1}{5} =  0.2 ~cylcle/pixel$

\subsection{Coordonn�es de la raie maximale}

Voici le code correspondant � l'affichage des coordonn�es :
\begin{Verbatim}[commandchars=\\\{\}]
max_1 = 0; 
i_max_1 = 0;
j_max_1 = 0;

\codeBlue{// Get the primary and secondary ray position}
for (j=0; j<H; j++) 
\{
	for (i=0; i<W; i++) 
	\{
		p = getPixel(i,j);
		
		\codeBlue{// get the primary ray}
		if ( max_1 < p)
		\{
			max_1 = p;
			i_max_1 = i;
			j_max_1 =j;
		\} 
	\}
\}

\codeBlue{// Printing the value of the points}
print ("i_max_1 = ", i_max_1, ", j_max_1 = ", j_max_1);
\end{Verbatim}


Les valeurs de \emph{i\_max\_1} et \emph{j\_max\_1} correspondants aux coordonn�es de la raie maximale dans le plan de Fourier valent apr�s ex�cution du code pr�c�dent : (128,128)

\subsection{Affichage de la raie secondaire ainsi que la distance la s�parant de la raie principale}

Voici le code correspondant :
\begin{Verbatim}[commandchars=\\\{\}]
max_1 = 0; 
i_max_1 = 0;
j_max_1 = 0;

max_2 = 0; 
i_max_2 = 0;
j_max_2 = 0;

\codeBlue{// Get the primary and secondary ray position}
for (j=0; j<H; j++) 
\{
	for (i=0; i<W; i++) 
	\{
		p = getPixel(i,j);
		
		\codeBlue{// get the primary ray}
		if ( max_1 < p)
		\{
			max_1 = p;
			i_max_1 = i;
			j_max_1 =j;
		\} 

		\codeBlue{// get the secondary ray}
		if ( max_2 < p && max_1 > p)
		\{
			max_2 =p;
			i_max_2 = i;
			j_max_2 =j;
		\}
	\}
\}

\codeBlue{// Calculating the distance between the two ray}
distance = sqrt(pow(i_max_2 - i_max_1, 2) + pow(j_max_2 - j_max_1, 2));

\codeBlue{// Printing the value of the points and the distance between them}
print ("i_max_1 = ", i_max_1, ", j_max_1 = ", j_max_1);
print ("i_max_2 = ", i_max_2, ", j_max_2 = ", j_max_2);
print ("D = ", distance);
\end{Verbatim}

\subsection{Tableau regroupant les informations (Direction, $\omega$, $P_x$, etc) des images de taille 128x128}

\begin{tabular}{|c|c|c|c|c|c|c|}
\hline
Image & Direction & T  & $\omega$ &$P_x$ &$P_y$ & D  \\
 & (x,y,xy) &(pixel dans l'image) & (cycle/pixel)  & (dans fourier) & (dans fourier) & (dans fourier) \\
 \hline
128a & verticale & 5 & 0.2 & 64 & 90 & 26 \\
\hline
128b & diagonale & 1.77 & 0.57 & 115 & 115 & 72 \\
\hline
128c & horizontale & 5 & 0.2 & 90 & 64 & 26 \\
\hline
128d & verticale & 10 & 0.1 & 64 & 77 & 13 \\
\hline
128e & diagonale & 7.14 & 0.14 & 77 & 77 & 18 \\
\hline
128f & horizontale & 10 & 0.1 & 77 & 64 & 13 \\
\hline
\end{tabular}


\begin{paragraph} {Pour ces images 128x128 on a:}~\\
$\omega_a$ = 26 / 128 = 0.2 \\
$\omega_c$ = 26 / 128 = 0.2 \\
$\omega_d$ = 13 / 128 = 0.1 \\
$\omega_f$ = 13 / 128 = 0.1 \\
\end{paragraph}

\subsection{Tableau regroupant les informations (Direction, $\omega$, $P_x$, etc) des images de taille 256x256}

\begin{tabular}{|c|c|c|c|c|c|c|}
\hline
Image & Direction & T  & $\omega$ &$P_x$ &$P_y$ & D  \\
 & (x,y,xy) &(pixel dans l'image) & (cycle/pixel)  & (dans fourier) & (dans fourier) & (dans fourier) \\
 \hline
256a & verticale & 5 & 0.2 & 128 & 179 & 51 \\
\hline
256b & diagonale & 3.6 & 0.28 & 179 & 179 & 72 \\
\hline
256c & horizontale & 5 & 0.2 & 179 & 128 & 51 \\
\hline
256d & verticale & 10 & 0.1 & 128 & 154 & 26 \\
\hline
256e & diagonale & 2.33 & 0.43 & 205 & 205 & 108.9 \\
\hline
256f & horizontale & 10 & 0.1 & 154 & 128 & 26 \\
\hline
\end{tabular}


\begin{paragraph} {Pour ces images 256x256 on a:}~\\
$\omega_a$ = 51 / 256 = 0.2 \\
$\omega_c$ = 51 / 256 = 0.2 \\
$\omega_d$ = 26 / 256 = 0.1 \\
$\omega_f$ = 26 / 256 = 0.1 \\
\end{paragraph}

\subsection{Tableau regroupant les informations (Direction, $\omega$, $P_x$, etc) des images de taille 512x512}

\begin{tabular}{|c|c|c|c|c|c|c|}
\hline
Image & Direction & T  & $\omega$ &$P_x$ &$P_y$ & D  \\
 & (x,y,xy) &(pixel dans l'image) & (cycle/pixel)  & (dans fourier) & (dans fourier) & (dans fourier) \\
 \hline
512a & verticale & 5 & 0.2 & 256 & 358 & 102 \\
\hline
512b & diagonale & 1.77 & 0.57 & 461 & 461 & 290 \\
\hline
512c & horizontale & 5 & 0.2 & 358 & 256 & 102 \\
\hline
512d & verticale & 10 & 0.1 & 256 & 307 & 51 \\
\hline
512e & diagonale & 7.14 & 0.14 & 307 & 307 & 72 \\
\hline
512f & horizontale & 10 & 0.1 & 307 & 256 & 51 \\
\hline
\end{tabular}


\begin{paragraph} {Pour ces images 512x512 on a:}~\\
$\omega_a$ = 102 / 512 = 0.2 \\
$\omega_c$ = 102 / 512 = 0.2 \\
$\omega_d$ = 51 / 512 = 0.1 \\
$\omega_f$ = 51 / 512 = 0.1 \\
\end{paragraph}


On se rend bien compte � la fin que la taille de l'image est en rapport avec $\omega$. On a la rapport suivant: \\
\\
$\omega~=~\frac{D}{\sqrt{taille\_image}}$

\newpage

\subsection{Classification de toutes les images en trois classes: horizontale, verticale, diagonale et calcule de la fr�quence spatiale pour chaque image}

Afin de classifier ces images il suffit de regarder les diff�rence sur la position de la raie principale et la raie secondaire. Si pour une image I c'est le x qui varie alors il faut plac� I dans la classe horizontale. Si c'est le y qui varie alors il faut plac� I dans la classe verticale. Et si c'est x et y qui varie alors il faut plac� I dans la classe diagonale. 

Et ensuite pour calculer la fr�quence spatiale il suffit d'utiliser la formule $\omega~=~\frac{D}{\sqrt{taille\_image}}$.\\
~

Voici le code correspondant:
\begin{Verbatim}[commandchars=\\\{\}]
\codeRed{macro} "direction FFT" \{
	
	\codeBlue{// The arrays to classify the texture of the images}
	classe_horizontale = newArray();
	classe_verticale = newArray();
	classe_diagonale = newArray();

	\codeBlue{// The arrays containing the spatial frequences}
	classe_horizontale_freq = newArray();
	classe_verticale_freq = newArray();
	classe_diagonale_freq = newArray();

	\codeBlue{// The current image name construction}
	image_name_letter = "";
	image_name_size = ""
	image_name = "";

	\codeBlue{// For all of the images}
	for (k=1; k<19; k++)
	\{
		\codeBlue{// Getting the size_name}
		if (k <= 6 )    \{ image_name_size = "128_"; \} 
		else if (k<=12) \{ image_name_size = "256_"; \}
		else            \{ image_name_size = "512_"; \}	
		
		\codeBlue{// Getting the letter_name}
		if (k % 6 == 0)      \{ image_name_letter = "f.jpg"; \}
		else if (k % 6 == 5) \{ image_name_letter = "e.jpg"; \}
		else if (k % 6 == 4) \{ image_name_letter = "d.jpg"; \}
		else if (k % 6 == 3) \{ image_name_letter = "c.jpg"; \}
		else if (k % 6 == 2) \{ image_name_letter = "b.jpg"; \}
		else if (k % 6 == 1) \{ image_name_letter = "a.jpg"; \}

		\codeBlue{// Opening the image}
		image_name = image_name_size + image_name_letter;
		open (image_name_size + image_name_letter);
		
		\codeBlue{// r�cup�ration du ID de l'image}
		image = getImageID();

		\codeBlue{// application de la FFT}
		run("FFT", "title='FFT de l'image courante'");

		\codeBlue{// r�cup�ration de ID de la FFT : du plan de fourier}
		fourier = getImageID();

		\codeBlue{// Getting the size of the fourier plan}
		W = getWidth();
		H = getHeight();
		
		max_1 = 0; 
		i_max_1 = 0;
		j_max_1 = 0;
	
		max_2 = 0; 
		i_max_2 = 0;
		j_max_2 = 0;

		\codeBlue{// Get the primary and secondary ray position}
		for (j=0; j<H; j++) 
		\{
			for (i=0; i<W; i++) 
			\{
				p = getPixel(i,j);
				
				\codeBlue{// get the primary ray}
				if ( max_1 < p)
				\{
					max_1 =p;
					i_max_1 = i;
					j_max_1 =j;
				\} 
		
				\codeBlue{// get the secondary ray}
				if ( max_2 < p && max_1 > p)
				\{
					max_2 =p;
					i_max_2 = i;
					j_max_2 =j;
				\}
			\}
		\}

		\codeBlue{// Calculating the distance between the primary ray and the second one}
		distance = sqrt(pow(i_max_2 - i_max_1, 2) + pow(j_max_2 - j_max_1, 2));

		\codeBlue{// Calculating the spatial frequency of the current image}
		omega = distance / sqrt(W*H);
		
		\codeBlue{// store the name of the image into the right array for classification}
		\codeBlue{// and the spacial frequencies}
		if (i_max_1 != i_max_2 && j_max_1 != j_max_2) 
		\{
			classe_diagonale = Array.concat(classe_diagonale, image_name);
			classe_diagonale_freq = Array.concat(classe_diagonale_freq, omega);
		\} 
		else if (i_max_1 != i_max_2) 
		\{
			classe_horizontale = Array.concat(classe_horizontale, image_name);
			classe_horizontale_freq = Array.concat(classe_horizontale_freq, omega);
		\} 
		else if (j_max_1 != j_max_2)
		\{
			classe_verticale = Array.concat(classe_verticale, image_name);
			classe_verticale_freq = Array.concat(classe_verticale_freq, omega);
		\}
	\}

	\codeBlue{// print the elements of the different classes}
	print("classe_diagonale : ");
	Array.print(classe_diagonale);
	Array.print(classe_diagonale_freq);

	print("classe_horizontale : ");
	Array.print(classe_horizontale);
	Array.print(classe_horizontale_freq);

	print("classe_verticale : ");
	Array.print(classe_verticale);
	Array.print(classe_verticale_freq);
\}
\end{Verbatim}




\end{document}










