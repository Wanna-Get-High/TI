 \documentclass[a4paper,10pt]{article}
\input{/Users/WannaGetHigh/workspace/latex/macros.tex}

\title{Projection perspective}
\author{Fran�ois \bsc{Lepan}}

\begin{document}
\maketitle

\section{Mod�les simples d'objets 3D}

\subsection{Un cube de c�t� 1 m dont le barycentre est l'origine du rep�re monde, et dont les c�t�s sont parall�les aux axes de ce rep�re}

Voici la fonction correspondant � la cr�ation de ce cube:
\begin{verbatimtab}
function m = cube()
	p1 = [-0.5, -0.5, -0.5, 1];
	p2 = [0.5, -0.5, -0.5, 1];
	p3 = [0.5, 0.5, -0.5, 1];
	p4 = [-0.5, 0.5, -0.5, 1];
	p5 = [-0.5, -0.5, 0.5, 1];
	p6 = [0.5, -0.5, 0.5, 1];
	p7 = [0.5, 0.5, 0.5, 1];
	p8 = [0.5, -0.5, 0.5, 1];
    
	m = [p1',p2',p3',p4',p5',p6',p7',p8']; 
endfunction
\end{verbatimtab}

\subsection{Une grille plane compos�e de 15 carr�s (5 selon x et 3 selon y) de c�t�s 1 m, dont le barycentre est l'origine du rep�re monde, et situ�e dans le plan z=0.}

Voici la fonction correspondant � la cr�ation de cette grille:
\begin{verbatim}
function m = grille()
    x = [1:6] - 3.5;
    y = [-1.5,-1.5,-1.5,-1.5,-1.5,-1.5];
    z = zeros(1,24);
    w = ones(1,24);

    m = [x, x, x, x; y, y+1,y+2, y+3; z; w]
endfunction
\end{verbatim}

\section{Matrice extrins�que}

\subsection{D�finition des quatre fonctions}

Voici les quatre fonctions qui permettent de calculer des matrices extrins�ques:

\begin{paragraph}{\emph{RotationX(theta)}}~\\
\begin{verbatim}
function m = RotationX(theta)
    m = [1,     0,        0,       0;
         0,cos(theta),-sin(theta), 0;
         0,sin(theta),cos(theta),  0;
         0,     0,        0,       1]
endfunction
\end{verbatim}
\end{paragraph}

\begin{paragraph}{\emph{RotationY(theta)}}~\\
\begin{verbatim}
function m = RotationY(theta)
    m = [cos(theta),0,sin(theta), 0;
         0,         0,      0,    0;   
         -sin(theta),0,cos(theta),0;
         0,         0,      0,    1]
endfunction
\end{verbatim}
\end{paragraph}

\begin{paragraph}{\emph{RotationZ(theta)}}~\\
\begin{verbatim}
function m = RotationZ(theta)
    m = [cos(theta),-sin(theta),0 ,0;
         sin(theta),cos(theta) ,0 ,0;   
         0         ,0          ,1 ,0;
         0         ,0          ,0 ,1]
endfunction
\end{verbatim}
\end{paragraph}

\begin{paragraph}{\emph{Translation(x,y,z)}}~\\
\begin{verbatim}
function m = translation(x,y,z)
    m = [1,0,0,x;
         0,1,0,y;
         0,0,1,z;
         0,0,0,1]
endfunction
\end{verbatim}
\end{paragraph}

\subsection{D�terminer les matrices extrins�ques positionnant les cam�ras suivantes:}

\subsubsection{Centre optique (0, 0, -5 m), axe optique orient� selon z, verticale de la cam�ra selon y}

Voici le code correspondant � la cr�ation de cette matrice:
\begin{verbatim}
?????????
\end{verbatim}


\subsubsection{Axe optique selon la diagonale principale du rep�re, regardant le centre du rep�re. Centre optique situ� � une distance de 5 m�tres du centre du rep�re. Verticale de la cam�ra dans un plan contenant z}

Voici le code correspondant � la cr�ation de cette matrice:
\begin{verbatim}
????????????
\end{verbatim}


\section{Matrice intrins�que}

Voici le calcule de la matrice intrins�que d�crit dans le sujet:

\begin{verbatimtab}
??
\end{verbatimtab}

+ en quoi le changement de la distance focale a elle une influence sur la sc�ne regarder.

\section{Projection et affichage des objets}

Repr�sentation des images obtenues par la cam�ra d�finie pr�c�demment:

\subsection{Pour la grille et le cube}

??????????????????????\\
??????????????????????\\
??????????????????????\\

\subsection{pour les deux positions d�finies par les matrices extrins�ques calcul�es}


??????????????????????\\
??????????????????????\\
??????????????????????\\


\end{document}